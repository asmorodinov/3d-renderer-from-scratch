% Это основная команда, с которой начинается любой \LaTeX-файл. Она отвечает за тип документа, с которым связаны основные правил оформления текста.
\documentclass[12pt]{article}
\usepackage{times}

% Здесь идет преамбула документа, тут пишутся команды, которые настраивают LaTeX окружение, подключаете внешние пакеты, определяете свои команды и окружения. В данном случае я это делаю в отдельных файлах, а тут подключаю эти файлы.

% Здесь я подключаю разные стилевые пакеты. Например возможности набирать особые символы или возможность компилировать русский текст. Подробное описание внутри.
\usepackage{packages}

%\usepackage[english]{babel}
%\addto{\captionsenglish}{%
%	\renewcommand{\bibname}{ref}
%}

% \renewcommand{\bibsection}{\subsection*{New title, typeset as un-numbered subsection.}}

% \renewcommand{\bibname}{References}

% Здесь я определяю разные окружения, например, теоремы, определения, замечания и так далее. У этих окружений разные стили оформления, кроме того, эти окружения могут быть нумерованными или нет. Все подробно объяснено внутри.
\usepackage{environments}

% Здесь я определяю разные команды, которых нет в LaTeX, но мне нужны, например, команда \tr для обозначения следа матрицы. Или я переопределяю LaTeX команды, которые работают не так, как мне хотелось бы. Типичный пример мнимая и вещественная часть комплексного числа \Im, \Re. В оригинале они выглядят не так, как мы привыкли. Кроме того, \Im еще используется и для обозначения образа линейного отображения. Подробнее описано внутри.
\usepackage{commands}

%
% Пакет для титульника исследовательского проекта
%\usepackage{titlepage}

% Здесь задаем параметры титульной страницы
%\setToProgram
%\setTitle{3D renderer с нуля}
%\setStage{}
%\setGroup{192}
%сюда можно воткнуть картинку подписи
%\setStudentSgn{\includegraphics[scale=0.2]{signature.png}}
%\setStudent{А.А.Смородинов}
%\setStudentDate{03.06.2021}
%\setAdvisor{Дмитрий Витальевич Трушин}
%\setAdvisorTitle{доцент, к.ф.-м.н.}
%\setAdvisorAffiliation{ФКН НИУ ВШЭ}
%\setAdvisorDate{}
%\setGrade{}
%сюда можно воткнуть картинку подписи
%\setAdvisorSgn{}
%\setYear{2021}


% С этого момента начинается текст документа
\begin{document}

% Эта команда создает титульную страницу
% \makeTitlePage

\title{План курсовой работы по теме "Продвинутый 3D renderer"}
% \author{Смородинов Александр, группа 196}
\date{30.01.2022}
\maketitle

% Здесь будет автоматически генерироваться содержание документа
\tableofcontents

\pagebreak

% Данное окружение оформляет аннотацию: краткое описание текста выделенным абзацем после заголовка
\section{Аннотация}
На сегодняшний день технологии отрисовки трёхмерных сцен используются во многих сферах нашей жизни: в 3d моделировании и анимации, в компьютерных играх и 3d/VR симуляциях. Первые статьи по данной теме были опубликованы ещё в 60х-70х годах 20 века, и с тех пор было проведено огромное количество исследований и работ в сфере 3d рендеринга. 

Цель данной работы - изучить и реализовать ряд важнейших алгоритмов 3d рендеринга. На втором курсе я реализовал базовую версию 3d рендера, в которой все основные алгоритмы уже были имплементированы. Поэтому в этой курсовой работе планируется продолжить изучать данную сферу, и реализовать более сложные техники отрисовки.

\section{Постановка задачи}
\subsection{Цели}
Основная цель проекта - это изучение и реализация алгоритмов, использующихся в компьютерной графике, на которых основаны большинство программ 3d моделирования, 3d игр, 3d/VR симуляторов и других приложений, имеющих какое-либо отношение к трёхмерной графике. 
\subsection{Задачи}
Основные задачи:
\begin{itemize}
	\item Изучить и реализовать алгоритмы (ниже будет конкретный список)
	\item Изложить теоретические основы этих алгоритмов, описать детали их реализации, написать сопроводительную документацию к коду и протестировать его
\end{itemize}

\subsection{Предварительный список техник, которые я планирую реализовать}
\begin{enumerate}
	\item Отображение нормалей (normal mapping)
	\item ''Параллакс'' отображение (parallax mapping)
	\item HDR
	\item Bloom
	\item Отложенное освещение и затенение (deferred shading) - в качестве эксперимента
	\item SSAO
	\item Физически корректный рендеринг (PBR)
	\item Поддержка прозрачных объектов
	\item Поддержка теней
	\item Геометрические шейдеры (может быть)
	\item Разные эффекты постобработки
	\item Сглаживание (antialiasing)
	\item Система частиц
\end{enumerate}

\subsection{Уже реализованные в прошлом году алгоритмы}
\begin{enumerate}
	\item Построение матрицы перехода из глобальной системы координат в пространство камеры
	\item Построение матриц проекции на экран (перспективная и ортогональная проекции)
	\item Удаление фрагментов треугольников, лежащих вне пирамиды зрения (view frustrum)
	\item Проверка точек на глубину с помощью z-буффера (при отрисовке объектов должны отрисовываться только ближайшие объекты к камере, но не те, которые находятся за ними)
	\item Отрисовка отрезков на экране (алгоритм Брезенхэма)
	\item Отрисовка треугольников на экране
	\item Перспективно правильная интерполяция параметров объектов при перспективной проекции
\end{enumerate}

\subsection {Репозиторий}
Репозиторий проекта: ~\cite{3dRenderer}

% Note also that very recently several constructions of~\cite{Elkik73} were clarified and simplified by Gabber and Ramero in~\cite[Chapter~5]{GabRam}.

\section{Актуальность и значимость}

% Этот пункт нужен только для программных проектов. В нем вы описываете, что у вас вообще должна быть за программа. На каком языке вы ее пишите. Что она должна делать. Подробное описание. Например, у вас пишется библиотека для работы с многочленами. Она должна предоставлять такие-то классы, пример использования. Она предоставляет такие-то методы, пример использования. Такая-то сложность методов. Например, еще можно написать, что вы тестируете библиотеку с помощью консольного приложения, данные считываются так-то, тестируются такие-то вещи, такие-то вещи выводятся. И так далее и тому подобное.

Акутальности темы 3d рендеринга в целом я уже касался выше, поэтому давайте поговорим про актуальность конкретно моей работы.

Важной особенностью моей реализации является то, что в моём проекте не используются такие известные API для работы с компьютерной графикой как OpenGL, DirectX, Vulcan и др. Это означает, что весь рендеринг происходит исключительно на CPU (так называемый software rendering). Данный подход имеет ряд своих преимуществ и недостатков, по сравнению с распространенным GPU rendering-ом:

Преимущества:
\begin{itemize}
	\item Работает на всех устройствах, независимо от наличия видеокарты. (в том числе микроконтроллерах и других встроенных системах)
	\item Также не нужно требовать от видеокарты пользователя поддержки конкретной версии API (например opengl 3.3)
	\item На всех устройствах одно и то же приложение работает одинаково, нет уязвимости к багам реализации API.
	\item С точки зрения программирования software renderer-а, есть полный контроль над реализацией. Это означает, что можно добавлять свои произвольные методы в пайплайн отрисовки, или изменять какие-то части пайплайна. 
	\item С образовательной точки зрения, самостоятельная реализация 3d рендерера с нуля даёт намного более глубокое понимание работы алгоритмов 3d рендеринга, чем просто использование некоторой готовой библиотеки.
\end{itemize}

Недостатки:
\begin{itemize}
	\item Главный недостаток - скорость. Видеокарты намного лучше справляются с множеством хорошо распараллеливаемых вычислений, чем процессоры. Они специально для этого и проектировались. Разница в скорости очень сильно зависит от конкретной видеокарты и конкретного приложения, но может составлять несколько порядков.
	\item С точки зрения программиста - нужно всё писать с нуля самому, а не использовать API (хотя также есть и уже написанные библиотеки).
\end{itemize}

Как можно видеть из данных особенностей, моё приложение вряд ли будет иметь шанс соперничать по скорости с библиотеками, использующими мощности видеокарт. 

Тем не менее, в случаях, когда скорость не так важна, но необходима кроссплатформенность и отсутствие большого количества внешних зависимостей, я надеюсь, что мой 3d рендерер может быть полезен в качестве библиотеки отрисовки.

Также, данный проект возможно будет полезен для некоторых людей, так же как и я интересующихся 3d графикой, как пример реализации software 3d рендерера на более-менее современном стандарте c++17.

Ну и очевидно, работа также значима лично для меня, как возможность изучить многие алгоритмы 3d рендеринга и реализовать их на практике.

\section{Существующие работы и решения}
Не знаю, сколько точно существует работ по данной теме, но я уверен, что их количество измеряется в сотнях, если не в тысячах. Очевидно, что все аналоги я не смогу здесь привести и описать, но постараюсь хотя бы упомянуть самые популярные. Проекты я брал из списка на github ~\cite{analogsList}.

\begin{enumerate}
	\item ssloy/tinyrenderer ~\cite{renderer1} Отличный проект, весь код занимает примерно 500 строчек. Насколько я понимаю, реализовано не так много алгоритмов 3d рендеринга (например нет клиппинга). Вообще проект в основном носит образовательный характер, и в этом плане там очень хорошие уроки на вики странице.
	\item zauonlok/renderer ~\cite{renderer2} Тоже интересный проект, написан на C89, реализовано довольно много вещей, но более продвинутые техники также не реализованы (которые я планирую реализовать).
	\item kosua20/herebedragons ~\cite{renderer3} Не совсем корректно считать этот проект аналогом моего, так как это не software рендерер, а наоборот, реализация одной сцены с помощью разных графических API. Интересно то, что также есть реализации сцены на платформах, где никаких API для 3d графики нет (например для PICO-8 и Nintendo Game Boy Advance). На таких платформах используются различные способы обойти аппаратные ограничения.
	\item skywind3000/mini3d ~\cite{renderer4} Небольшой (700 строк) 3d рендерер на c. Почти нет никаких реализованных дополнительных/продвинутых алгоритмов.
	\item ssloy/tinyraycaster ~\cite{renderer5} Как видно из названия, это рейкастер, а не полноценный 3d рендерер, поэтому тоже не очень корректный пример.
	\item skywind3000/RenderHelp ~\cite{renderer6} Ещё один довольно простой 3d рендерер на C++.
	\item Angelo1211/SoftwareRenderer ~\cite{renderer7} Один из самых интересных проектов из всего списка. Реализовано довольно много продвинутых алгоритмов. Тем не менее, есть несколько проблем с сглаживанием и ''Муаровым эффектом'' в некоторых сценах, а также есть нереализованные алгоритмы (из моего планируемого списка).
	\item martinResearch/DEODR ~\cite{renderer8} Дифференцируемый 3d рендерер на c. Какая-то классная штука для ML. Не уверен, часто ли данную библиотеку используют в других областях.
\end{enumerate}

Бонусные аналоги (не из списка выше, по крайней мере не из топа по звёздам)
\begin{enumerate}
	\item bytecode77/fastpix3d ~\cite{renderer9} Пока сильно не разбирался как именно, но данный 3d рендерер очень хорошо оптимизирован, и действительно показывает хорошие показатели по производительности. Хотя в примерах используются не очень детализированные модели и текстуры в не очень высоком разрешении, производительность не может не впечатлять, учитывая, что это software рендерер. Каких-то очень продвинутых алгоритмов не реализовано, хотя есть поддержка освещения и теней.
	\item Dawoodoz/DFPSR ~\cite{renderer10} Мне очень нравятся идеи, применённые в данном проекте. Библиотека предназначена для изометрических игр/сцен, и если использовать тот факт, что объекты всегда будут видны только с одного ракурса, то можно ''запечь''(pre-render, bake) 3d модель в три текстуры - diffuse, normal, height, а дальше работать с моделью как с одним прямоугольником (причём ориентированным в пространстве камеры вдоль координатных осей). Это очень сильно снижает необходимое количество вычислений на процессоре и позволяет отрисовывать достаточно сложные изометрические сцены в реальном времени.
\end{enumerate}

Как можно видеть, довольно мало проектов реализуют какие-то продвинутые алгоритмы 3d рендеринга, либо же я плохо искал. Это немного обнадёживает в плане того, что работа кажется довольно актуальной (так как аналогов не так много).

\section{Предлагаемые подходы и методы}
В основном используются достаточно классические алгоритмы, описанные во многих книгах и статьях, например ~\cite{Math3d}, ~\cite{LRN}. Некоторые алгоритмы, которые уже были реализованы в прошлом году, я указал в секции 2.4.

Я бы не сказал, что подходы, которые я использую, обладают большой новизной, но разумеется реализации даже одно и того же алгоритма разными людьми может часто выглядеть довольно по разному. Также, я стараюсь писать в более-менее современном стиле c++17, и каких-то очень похожих реализаций 3d рендерера я пока что не видел.

\section{Ожидаемые результаты}
Кроссплатформенное интерактивное приложение, в котором можно переключаться между сценами, редактировать сцены и отдельные объекты, менять различные параметры освещения и наглядно наблюдать разные алгоритмы отрисовки 3d моделей. 

Также есть отдельное ''ядро'' 3d renderer-а, используя которое в качестве библиотеки можно написать свою произвольную программу, для которой необходима отрисовка 3d сцен.

Ещё ожидается, что весь код будет протестирован, задокументирован и содержать комментарии.

\section{План работ}
Примерный план:
\begin{itemize}

\item 30.01.2022 - план КР, normal mapping, parallax mapping, небольшие оптимизации 3d renderer-а (уже сделано)
\item 01.03.2022 - HDR, Bloom, SSAO, PBR
\item 01.04.2022 - Прозрачные объекты, постобработка, сглаживание, система частиц
\item 01.05.2022 - Тени, отложенный рендеринг, может быть что-то ещё
\item По мере работы - Оптимизация производительности и улучшение качества кода

\end{itemize}

\section{Список источников}

\begin{itemize}
\item learnopengl.com ~\cite{learnopengl}
\item Mathmatics for 3d game programming and computer graphics ~\cite{Math3d}
\item A Mathematical Introduction with OpenGL ~\cite{LRN}
\end{itemize}

\renewcommand{\refname}{}

\bibliographystyle{plainurl}
\bibliography{bibl}



% С этого момента глобальная нумерация идет буквами. Этот раздел я добавил лишь для демонстрации возможностей LaTeX, его можно и нужно удалить и он не нужен для курсового проекта непосредственно.
%\appendix

%Проведем небольшой обзор возможностей \LaTeX. Далее идет обзорный кусок, который надо будет вырезать. Он приведен лишь для демонстрации возможностей \LaTeX.

%\section{Нумеруемый заголовок}
%Текст раздела
%\subsection{Нумеруемый подзаголовок}
%Текст подраздела
%\subsubsection{Нумеруемый подподзаголовок}
%Текст подподраздела

%\section*{Не нумеруемый заголовок}
%Текст раздела
%\subsection*{Не нумеруемый подзаголовок}
%Текст подраздела
%\subsubsection*{Не нумеруемый подподзаголовок}
%Текст подподраздела


%\paragraph{Заголовок абзаца} Текст абзаца

%Формулы в тексте набирают так $x = e^{\pi i}\sqrt{\text{формула}}$. Выключенные не нумерованные формулы набираются либо так:
%\[
%x = e^{\pi i}\sqrt{\text{формула}}
%\]
%Либо так
%$$
%x = e^{\pi i}\sqrt{\text{формула}}
%$$
%Первый способ предпочтительнее при подаче статей в журналы AMS, потому рекомендую привыкать к нему.

%Выключенные нумерованные формулы:
%\begin{equation}\label{Equation1}
% \label{имя-метки} эта команда ставит метку, на которую потом можно сослаться с помощью \ref{имя-метки}. Метки можно ставить на все объекты, у которых есть автоматические счетчики (номера разделов, подразделов, теорем, лемм, формул и т.д.
%x = e^{\pi i}\sqrt{\text{формула}}
%\end{equation}
%Или не нумерованная версия
%\begin{equation*}
%x = e^{\pi i}\sqrt{\text{формула}}
%\end{equation*}

%Уравнение~\ref{Equation1} радостно занумеровано.

%Лесенка для длинных формул
%\begin{multline}
%x = e^{\pi i}\sqrt{\text{очень очень очень длинная формула}}=\\
%\tr A - \sin(\text{еще одна очень очень длинная формула})=\\
%\cos z \Im \varphi(\text{и последняя длинная при длинная формула})
%\end{multline}

%Многострочная формула с центровкой
%\begin{gather}
%x = e^{\pi i}\sqrt{\text{очень очень очень длинная формула}}=\\
%\tr A - \sin(\text{еще одна очень очень длинная формула})=\\
%\cos z \Im \varphi(\text{и последняя длинная при длинная формула})
%\end{gather}

%Многострочная формула с ручным выравниванием. Выравнивание идет по знаку $\&$, который на печать не выводится.
%\begin{align}
%x = &e^{\pi i}\sqrt{\text{очень очень очень длинная формула}}=\\
%&\tr A - \sin(\text{еще одна очень очень длинная формула})=\\
%&\cos z \Im \varphi(\text{и последняя длинная при длинная формула})
%\end{align}

%\begin{theorem}
%Текст теоремы
%\end{theorem}
%\begin{proof}
%В специальном окружении оформляется доказательство.
%\end{proof}

%\begin{theorem}[Имя теоремы]
%Текст теоремы
%\end{theorem}
%\begin{proof}[Доказательство нашей теоремы]
%В специальном окружении оформляется доказательство.
%\end{proof}

%\begin{definition}
%Текст определения
%\end{definition}

%\begin{remark}
%Текст замечания
%\end{remark}

%\paragraph{Перечни:} Нумерованные
%\begin{enumerate}
%\item Первый
%\item Второй
%\begin{enumerate}
%\item Вложенный первый
%\item Вложенный второй
%\end{enumerate}
%\end{enumerate}

%Не нумерованные

%\begin{itemize}
%\item Первый
%\item Второй
%\begin{itemize}
%\item Вложенный первый
%\item Вложенный второй
%\end{itemize}
%\end{itemize}


% Здесь текст документа заканчивается
\end{document}
% Начиная с этого момента весь текст LaTeX игнорирует, можете вставлять любую абракадабру.
